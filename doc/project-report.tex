\documentclass[11pt]{article}
\usepackage{cite}
\usepackage{lmodern}
\usepackage[utf8]{inputenc}
\usepackage[finnish]{babel}
\usepackage{hyperref}

\title{Network Analysis - Project Report}
\author{
    Jimi Hytönen\\
    \and Hanna Holtdirk\\
    \and Basil Mashal
}
\begin{document}

\maketitle

\section{Introduction}
This project is part of network analysis course. The purpose of this project was to analyse real-world network by applying different algorithms to extract some interesting information about the network as well as get hands-on experience with network analysis. We chose to analyse California road networks because TBD - why? 

TBD-something else?

In this report we will cover the technical aspects of gathering the data, tell how the work was divided, explain our network analysis and visualizations, and finally describe our conclusions. 



\section{Data}

We used dataset from https://www.cs.utah.edu/~lifeifei/SpatialDataset.html which was collected, cleaned and formated from multiple different sources into easy-to-use format. Network's nodes were in longitude-latitude coordinate form and network's edges contained information about start node, end node and the distance between them. In addition, the site provided information about California's points of interests such as hospitals, lakes and airports.

We used pandas for data manipulation and processing as we needed to get the data into a certain form for further analysis. We used networkx for building and visualisation of the network. This was really straightforward to do as the data was in an easy to use format.  TBD-something else?

\newpage

\section{Methods}
For the project we divided the original question into the following subtasks: 
\begin{enumerate}
\item What is the general structure of the road network
\item Can we find where the (big) cities are from the road network and points of interest
\item Can we learn to make predictions about the placement of the roads and places of interest
\end{enumerate}

For each of these subtasks we planned what analysis was needed to answer the question. After setting these subtasks we created a timeline for the project. 

For the first question we analysed the California road network with simple methods to get a better idea what we are dealing with. One of which was calculating the connectivity of the network to determine if there were roads that led nowhere. We also calculated and visualized different centrality measures of the network such as degree centrality, betweenness centrality, eigenvector centrality and katz centrality. 

After analysing the general structure of the network, we moved to the second question. For finding the big cities we used Girvan Newman community detection algorithm as it seemed to be the most suitable for the task. However, we didn't use the algorithm on the whole network because it was enormous, instead we used approximation based on previously calculated centrality measures to create a smaller graph and applied the algorithm on that.

For the third question ...



\section{Results}





\section{Conclusions}










\section{Contributors}


\end{document}
